\section{Hardware Design}
\begin{figure}[h]
	\centering
	\includegraphics[width=\linewidth,angle=0]{./Artwork/System.png}
	\caption{Morfeas RPI Hat's Block diagram}
	\label{fig:block}
\end{figure}

At figure~\ref{fig:block} present the design of the Morfeas RPI Hat in a simplified block diagram. The connection between the Raspberry PI and the Morfeas RPI hat done
from the RPi pin header that provided on the Raspberry board for expansion. From the pins that provided only the $I^2C$, UART, 5V power and some GPIO are used.
The 5V Power pins used to feedback with power the Raspberry Pi computer, for this a dedicated step down converter where is located onboard,
in addition the Power supply is supplying with power all the circuits that exist on board.

The UART pins driven directly to a dedicated LVTTL to RS-232 converter and from this directly a female pinhead. From there it's can be connected to any kind of connector
(like DB-9).

The $I^2C$ bus is the main communication way between the device onboard and the Raspberry pi computer. The device that communicating by this described bellow.
\begin{itemize}
	\item Current Sense Amplifier(CSA), one for each Port.
	\item Configuration serial EEPROM.
	\item Real Time Clock.
\end{itemize}

The CSAs are implemented by the MAX9611 which is a integrated circuit with main functionality to provide current measurements by reading the voltage of a shunt resistor on the high supple side.
In addition, the same IC can digitize the measurements of port's current, Output voltage and it's own temperature. All of them can be read by the $I^2C$ Bus.
Also the same IC can used as a electronic fuse in linear or switching mode.
For the implementation of the Morfeas RPI Hat the MAX9611 ICs are configured to work in the switching mode.
The MAX9611 are controlling P-Channel MOSfets that are switching ON or OFF the power of the port.
The threshold point is programmed by two resistors that make a voltage divider. The threshold limit for each Channel is approximately 4 Amperes.\\
More information for this can be found at the MAX9611's Datasheet at the ``Datasheets" directory inside the ``Hardware" folder.

The Configuration EEPROM (24AA08) give or get the calibration table for each CSA on request of the Morfeas system, or the Morfeas\_RPI\_Hat software.
This EEPROM have four individual memory pages with 2kbit in each.
For the current implementation of the Morfeas RPI Hat only two of them are used to store the calibration tables.

The RTC is an IC that keeping track of the absolute time. In general provide the functionality on a computer to can have the absolute time in every request.
The RTCs must have a battery as second supply that will allow them to count even if the main supply is off power.
Some RTCs required to have an external crystal resonator where use as part of the feedback filter of Pierce oscillator.
The frequency of the oscillator is derived from the characteristics of the filter and because is crystal based it have very good stability ($\pm3.5ppm$).
The output of the oscillator is used as a clock signal for a programmable counter where is counting the time.
The IC (DS3231) that have been chosen for this design have all the part of the oscillator (including the filter) integrated.\\
The communication with this IC done by $I^2C$ bus. All the operate with the RTC is handled by the kernel of the GNU operating system.

In addition to all the above the Morfeas RPI Hat using also some extra GPIO pins from the RPI Header that driving a RGB LED.
This have purpose to provide an indication of the mode of the Morfeas System (Morfeas\_SDAQ\_if).


